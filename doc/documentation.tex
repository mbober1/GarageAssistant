\documentclass[12pt]{article}
\linespread{1.3}
\usepackage[T1]{fontenc}
\usepackage{polski}
\usepackage[utf8]{inputenc}
\usepackage{geometry}
\usepackage{graphicx}
\usepackage{enumitem}
\usepackage{textcomp}
\usepackage{hyperref}
\usepackage{subcaption}
\usepackage{caption}
\usepackage{graphicx} 
\usepackage{float}
\usepackage{xurl}
\usepackage{xcolor}

\usepackage{lscape}

\newgeometry{tmargin=3cm, bmargin=3cm, lmargin=3cm, rmargin=3cm}

\begin{document}

\begin{titlepage}
	\center
	
	
	%--------------  NAGŁÓWKI ---------------------------------------------------------
	
	\textsc{\Large Politechnika Wrocławska}\\[0.5cm] % Name of your university/college
	\textsc{\large Wydział Elektroniki}\\[5cm] % Major heading such as course name
	
	%--------------  TYTUŁY   ---------------------------------------------------------
	
	\textsc{\LARGE PROJEKT ZESPOŁOWY}\\[2cm] % Minor heading such as course title
	{ \Large \bfseries GarageAssistant}\\[4cm] % Title of your document
	%--------------  TERMIN ZAJĘĆ -----------------------------------------------------
	\begin{minipage}{0.83\textwidth}
	\begin{flushleft} \large
	Termin zajęć: Środa, 13-16 \\[2cm]
	\end{flushleft}
	\end{minipage}
	\\[2cm]
	
	%--------------  AUTORZY  ---------------------------------------------------------
	
	\begin{minipage}{0.4\textwidth}
	\begin{flushleft} \large
	\emph{Autorzy:}\\
	Marcin \textsc{Bober}\\
	Janusz \textsc{Domaradzki}\\
	Michał \textsc{Kowalski}\\
	\end{flushleft}
	\end{minipage}
	~
	\begin{minipage}{0.4\textwidth}
	\begin{flushright} \large
	\emph{Prowadzący zajęcia:} \\
	dr inż. Krzysztof \textsc{Arent} 
	\end{flushright}
	\end{minipage}\\[1cm]
	
	%--------------  DATA     ----------------------------------------------------------
	
	{\large Wrocław, \today}\\[1cm]
	
	\vfill 
	\end{titlepage}

\tableofcontents

\clearpage
\section{Problem projektu}
W dzisiejszych czasach wysoka urbanizacja i ciągłe zagęszczanie zabudowań 
mają negatywny wpływ na ciągle kurczące sie powierzchnie miejsc parkingowych.
Nie pomaga także fakt znacznej popularyzacji samochodów typu SUV, które cechują się
większymi wymiarami w porównaniu do tradycyjnego samochodu miejskiego. 
Z tego powodu możemy zaobserować nasilający się problem parkowania dużych aut
w ciasnych zaułkach. Często są to zadania na tyle karkołomne że użytkownicy 
zniechęcają się do poruszania przy użyciu aut i w szczególności w dużych aglomeracjach
przesiadają się do komunikacji miejskiej. Bardzo często jest to dobre rozwiązanie
dla naszej planety niemniej jednak nie powinno powodować to niechęci wobec samochodów osobowych.
Niechęci te są wyszczególniane przedewszystkim w okresach takich jak ten,
gdy musimy egzystować w świecie ogarniętym pandemią wirusa i bezpieczniej jest
podrózować osobistym środkiem transporu. Z tego też powodu wychodzimy naprzeciw
potrzeb naszych klientów z propozycją systemu wspomagającego kierowcę podczas manewru
parkowania. Uznaliśmy że czynnoś ta jest szczególnie ważną że względu na to że znaczącą
częścią incydentów z użyciem pojazdów są stłuczki parkingowe (TUTAJ ŹRÓDŁO JAKIŚ BADAŃ).

\section{Plan pracy i rozkład w czasie}
\newpage
\section{Doręcznie}
\section{Budżet}
\newpage
\section{Zarządzanie projektem}
\section{Zespół}

\end{document}
